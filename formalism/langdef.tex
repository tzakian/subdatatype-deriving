\section{Language Definition}
\label{sec:langdef}

\begin{itemize}
  \item Why does this not create open datatypes? How does it differ and how
  does it get around the problems found in open datatypes?
 \item Currently only have core $F_C(X)$ as presented in~\cite{systemFC}.
 Need to extend this with the extendable data type section (i.e., the hard
 stuff...).
\end{itemize}

{\begin{figure}[H]
\begin{minipage}[t]{0.5\linewidth}
\vspace{5mm}
\fbox{{\bf Symbol Classes}}
\vspace{2mm}

$
\begin{array}{lcl}
  a,b,c,co & \to & \langle \textrm{type variable} \rangle \\
  x,f & \to & \langle \textrm{term variable} \rangle \\
  C & \to & \langle \textrm{coercion constant} \rangle \\
  T & \to & \langle \textrm{value type constructor} \rangle \\
  S_n & \to & \langle n-\textrm{ary type function} \rangle \\
  K & \to & \langle \textrm{data constructor} \rangle
\end{array}
$

\vspace{5mm}
\fbox{{\bf Sorts and Kinds}}
\vspace{2mm}

$
\begin{array}{lcll}
  \delta & \to & \textrm{TY} \alt \textrm{CO} & \textrm{Sorts} \\
  \kappa, \iota & \to & 
    \star \alt \kappa_1 \to \kappa_2 \alt \sigma_1 \sim \sigma_2 & \textrm{Kinds}
\end{array}
$

\vspace{5mm}
\fbox{{\bf Syntactic sugar}}
\vspace{2mm}

$
\textrm{Types} \;\;\; \kappa \Rightarrow \sigma \; \equiv \; \forall \_ :\kappa. \sigma
$

\end{minipage}
\begin{minipage}[t]{0.5\linewidth}
\vspace{5mm}
\fbox{{\bf Declarations }}
\vspace{2mm}

$
\begin{array}{lcl}
  pgm & \to & \overline{cdecl}; \overline{decl}; e \\
  cdecl & \to & \mathbf{data}\;\mathbf{extendable}\;{T\; : \; \overline{\kappa} \to \star \;\mathbf{where}\;} \\
  & & \;\; \overline{{\tt K} : \forall\; \overline{a:\kappa}.\;
  \forall\;\overline{b : \iota}. \overline{\sigma} \to T\; \overline{a}
       \;\;\;}  \\
  & \alt & \mathbf{data}\;{T\; : \; \overline{\kappa} \to \star \;\mathbf{where}\;} \\
  & & \;\; \overline{{\tt K} : \forall\; \overline{a:\kappa}.\;
  \forall\;\overline{b : \iota}. \overline{\sigma} \to T\; \overline{a}
       \;\;\;} \\
  & & \mathbf{extends}(\overline{T})\\
  & \alt & \mathbf{data}\;\mathbf{extendable}\;{T\; : \; \overline{\kappa} \to \star \;\mathbf{where}\;} \\
  & & \;\; \overline{{\tt K} : \forall\; \overline{a:\kappa}.\;
  \forall\;\overline{b : \iota}. \overline{\sigma} \to T\; \overline{a}
       \;\;\;} \\
  & & \mathbf{extends}(\overline{T})\\
  decl & \to & \mathbf{data}\;{T\; : \; \overline{\kappa} \to \star \;\mathbf{where}\;} \\
  & & \;\; \overline{{\tt K} : \forall\; \overline{a:\kappa}.\;
  \forall\;\overline{b : \iota}. \overline{\sigma} \to T\; \overline{a}
       \;\;\;}  \\
  & \alt & \mathbf{type}\; S_n ~:~ \overline{\kappa}^n \to \iota \\
  & \alt & \mathbf{axiom}\; C ~:~ \sigma_1 \sim \sigma_2
\end{array}
$
\end{minipage}


\begin{minipage}[t]{0.45\linewidth}
\vspace{5mm}
\fbox{{\bf Types and Coercions}}
\vspace{2mm}

$
\begin{array}{lcl}
  d & \to & a \alt T  \hspace{10mm} \textrm{Atom of sort TY}\\
  g & \to & c \alt C  \hspace{10mm}  \textrm{Atom of sort CO}\\
\varphi, \rho, \sigma, && \\
\tau, \nu, \gamma &
  \to & a \alt C \alt T \alt \varphi_1 ~ \varphi_2 \alt S_n ~
  \overline{\varphi}^n \alt \forall a:\kappa.\varphi \\
  & \alt & \textrm{sym} ~ \gamma \alt \gamma_1 \circ \gamma_2 \alt
  \gamma@\varphi \alt \textrm{left} ~ \gamma \alt \textrm{right} ~ \gamma \\
  & \alt & \gamma \sim \gamma \alt \textrm{rightc} ~ \gamma \alt
  \textrm{leftc} ~ \gamma \alt \gamma \blacktriangleright \gamma \\
  & \alt & \zeta_1 \tstack{\sim}{[T_1, T_2]} \zeta_2 \\
    %\alt \zeta_1 \tstack{\triangleright}{[T_1, T_2]} \zeta_2\\
  \zeta & \to & (K, \sigma) \alt \sigma
\end{array}
$
%\underset{\text{below}}{\overset{\text{above}}{+}}
\end{minipage}
\begin{minipage}[t]{0.5\linewidth}
\vspace{5mm}
\fbox{{\bf Terms}}
\vspace{2mm}

$
\begin{array}{lcll}
  u & \to & x \alt K & \textrm{Variables and data consructors} \\
  e & \to & u & \textrm{Term atoms} \\
    & \alt & \Lambda a:\kappa.e \alt e \varphi & \textrm{Type abstractions/application} \\
    & \alt & \lambda x : \sigma.e \alt e_1 ~ e_2 & \textrm{Term abstraction/application} \\
    & \alt & \letin{x:\sigma}{e_1}{e_2} \\
    & \alt & \caseof{e_1} {\overline{p \to e_2}} \\
    & \alt & e \blacktriangleright \gamma
      \alt   e \tstack{\triangleright}{[T_1,T_2]} \gamma & \textrm{Cast}\\
    \\
p & \to & K \; \overline{b:\kappa} \; \overline{x:\sigma} & \textrm{Pattern}
\end{array}
$
\end{minipage}

\vspace{5mm}
\fbox{{\bf Environments}}
\vspace{2mm}

$
\Gamma \;\;\; \to \;\;\; \epsilon \alt \Gamma,u:\sigma \alt \Gamma,d:\kappa
\alt \Gamma,g:\kappa \alt \Gamma,S_n:\kappa
\alt \Gamma, (T:\overline{\zeta})
$

A top-level environment binds only type constructors, \\
$T,S_n$, data constructors $K$, and coercion constants $C$.
\caption{The core language for extensible data types -- $F_C(X)$}
\label{fig:langdef}
\end{figure}}

