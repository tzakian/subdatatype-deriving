\section{Translation}
\label{sec:translation}
In this section we go about detailing the source translation in order to
implement these rules and to allow the use of extendable data types in
Haskell.
\subsubsection{Insertion of Casts}

In order to insert the casts correctly, we must do coverage analysis of the
matching on the data types e.g.,
\begin{lstlisting}
f :: T1 -> T2
f ...
\end{lstlisting}
In this case we perform coverage analysis on {\tt T1} and, when {\tt f} fails to
match on all of the data types, automatically inserts a default clause (if
there is not already one) which behaves like the identity function, except
that it inserts a cast $(K,\sigma) \tstack{\triangleright}{T_1,T_2}$. We
now formalise this argument.

Define the set of all constructors of an extended data type $T$ 
coming from extended data types as $C$. Define the set of all specific
constructors defined by $T$ as $D$. Then we have that all of the
constructors of $T$ are $C \sqcup D$.

Now when performing coverage analysis of {\tt f} we consider two cases --
letting $R$ be the covered cases and $R'$ the non-covered cases:
\begin{itemize}
  \item $R = C \sqcup D$: In this case, the programmer has manually
    coded a conversion between the two data types. Therefore, we do not
    have to insert any casts.
  \item $R \supsetneq D$. Then we add a default clause to {\tt f} that performs
    a type coercion on the input.
  \item $R \subsetneq D$. In this case, it is an error, and we do not
    insert casts, since we cannot add a catchall clause to {\tt f}.
  \item {\bf It is
    worth noting that we could relax this requirement and add patterns for
  everything in $C \setminus R$ this would provide correctness right?}
\end{itemize}































































