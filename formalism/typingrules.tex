\section{Typing Rules}
\label{sec:tyrules}

\begin{figure}
  \begin{mathpar}
  \fbox{$\Gamma \vty \sigma : \kappa$}
  \\
\inference[TyVar]
  {(d : \kappa) \in \Gamma & \Gamma \vk \kappa : TY}
  {\Gamma \vty d : \kappa}
\and
\inference[TyApp]
  {\Gamma \vty \sigma_1 : \kappa_1 \to \kappa_2 & \Gamma \vty \sigma_2 : \kappa_1}
  {\Gamma \vty \sigma_1 ~ \sigma_2 : \kappa_2}
\\
\inference[TySCon]
  {(S_n : \overline{\kappa}^n \to \iota) \in \Gamma
    & \Gamma \vty \overline{\sigma : \kappa}^n}
  {\Gamma \vty S_n \overline{\sigma}^n : \iota}
\and
\inference[TyAll]
  {\Gamma,a:\kappa \vty \sigma : \star
   & \Gamma \vk \kappa : \delta
   & a \notin fv(\Gamma)}
  {\Gamma \vty \forall a:\kappa . \sigma : \star}
\\
\fbox{$\Gamma \vco \gamma : \sigma \sim \tau$}
\\
\inference[CoRefl]
  {(a:\kappa) \in \Gamma
   & \Gamma \vk \kappa : TY}
  {\Gamma \vco a : a \sim a}
\and
\inference[CoVar]
  {(g : \sigma \sim \tau) \in \Gamma}
  {\Gamma \vco g : \sigma \sim \tau}
\and
\inference[CoAllT]
  {\Gamma,a:\kappa \vco \gamma : \sigma \sim \tau \\
   \Gamma \vk \kappa : TY & a \notin fv(\Gamma)}
   {\Gamma \vco \forall a:\kappa.\gamma : \forall a:\kappa. \sigma \sim
   \forall a:\kappa. \tau}
\and
\inference[CoInstT]
  {\Gamma \vco \gamma : \forall a.\kappa.\sigma \sim \forall b:\kappa .\tau \\
  \Gamma \vty v : \kappa}
  {\Gamma \vco \gamma @ v : [v/a]\sigma \sim [v/b]\tau}
\and
\inference[SComp]
  {\Gamma \vco \overline{\gamma : \sigma \sim \tau}^n
   & \Gamma \vty S_n \overline{\sigma}^n : \kappa}
  {\Gamma \vco S_n \overline{\gamma}^n : S_n \overline{\sigma}^n \sim S_n
  \overline{\tau}^n}
\and
\inference[Sym]
  {\Gamma \vco \gamma : \sigma \sim \tau}
  {\Gamma \vco \gamma : \tau \sim \sigma}
\and
\inference[Trans]
  {\Gamma \vco \gamma_1 : \sigma_1 \sim \sigma_2 \\
   \Gamma \vco \gamma_2 : \sigma_2 \sim \sigma_3}
  {\Gamma \vco \gamma_1 \circ \gamma_2 : \sigma_1 \sim \sigma_3}
\and
\inference[Comp]
  {\Gamma \vco \gamma_1 : \sigma_1 \sim \tau_1
   & \Gamma \vco \gamma_2 : \sigma_2 \sim \tau_2
   \\ \Gamma \vty \sigma_1 ~ \sigma_2 : \kappa}
  {\Gamma \vco \gamma_1 ~ \gamma_2 : \sigma_1 ~\sigma_2 \sim \tau_1 ~ \tau_2}
\and
\inference[Left]
  {\Gamma \vco \gamma : \sigma_1 ~\sigma_2 \sim \tau_1 ~\tau_2}
  {\Gamma \vco \textrm{left} ~ \gamma : \sigma_1 \sim \tau_1}
\and
\inference[Right]
  {\Gamma \vco \gamma : \sigma_1 ~\sigma_2 \sim \tau_1 ~\tau_2}
  {\Gamma \vco \textrm{right} ~ \gamma : \sigma_1 \sim \tau_1}
\and
\inference[CompC]
  {\Gamma \vco \gamma : \kappa_1 \sim \kappa_2
   & \Gamma \vco \gamma' : \sigma_1 \sim \sigma_2
   \\ \Gamma \vk \kappa_1 : CO}
  {\Gamma \vco \gamma \Rightarrow \gamma' : (\kappa_1 \Rightarrow \sigma_1)
  \sim (\kappa_2 \Rightarrow \sigma_2)}
\and
\inference[LeftC] % TODO: stack the two coercions
  {\Gamma \vco \gamma :
    \kappa_1 \Rightarrow \sigma_1
    \sim
    \kappa_2 \Rightarrow \sigma_2}
  {\Gamma \vco \textrm{leftc} ~ \gamma : \kappa_1 \sim \kappa_2}
\and
\inference[RightC]
  {\Gamma \vco \gamma :
    \kappa_1 \Rightarrow \sigma_1
    \sim
    \kappa_2 \Rightarrow \sigma_2}
  {\Gamma \vco \textrm{rightc} ~ \gamma : \sigma_1\sim \sigma_2}
\and
\inference[$(\sim)$]
  {\Gamma \vco \gamma_1 : \sigma_1 \sim \tau_1
   & \Gamma \vco \gamma_2 : \sigma_2 \sim \tau_2}
  {\Gamma \vco \gamma_1 \sim \gamma_2 : (\sigma_1 \sim \sigma_2) \sim (\tau_1 \sim \tau_2)}
\and
\inference[CastC]
  {\Gamma \vco \gamma_1 : \kappa
   & \Gamma \vco \gamma_2 \kappa \sim \kappa'}
   {\Gamma \vco \gamma_1 \blacktriangleright \gamma_2}
\\
\fbox{$\Gamma \ve e : \zeta$}
\\
\inference[Var]
  {(u:\sigma) \in \Gamma}
  {\Gamma \ve u : \sigma}
\and
\inference[Case]
  {\Gamma \ve e : \sigma
   & \overline{\Gamma \vp p \to e : \sigma \to \tau}}
  {\Gamma \ve \caseof{e}{\overline{p \to e}} : \tau}
\and
\inference[Let]
  {\Gamma \ve e_1 : \sigma_1
   & \Gamma , x:\sigma_1 \ve e_2 : \sigma_2}
  {\Gamma \ve \letin{x:\sigma_1}{e_1}{e_2} : \sigma_2}
\and
\inference[Cast]
  {\Gamma \ve e : \sigma
   & \Gamma \vco \gamma : \sigma \sim \tau}
  {\Gamma \ve e \blacktriangleright \gamma : \tau}
\and
\inference[Abs]
  {\Gamma \vty \sigma_x : \star \\
   \Gamma ,x:\sigma_x \ve e : \sigma}
  {\Gamma \ve \lambda x:\sigma_x . e : \sigma_x \to \sigma}
\and
\inference[App]
  {\Gamma \ve e_1 : \sigma_2 \to \sigma_1 \\
   \Gamma \ve e2 : \sigma_2}
  {\Gamma \ve e_1 ~ e_2 : \sigma_1}
\and
\inference[AbsT]
  {\Gamma a : \kappa \ve e : \sigma 
   & \Gamma \vk \kappa : \delta
   & a \notin fv(\Gamma)}
  {\Gamma \ve \Lambda a : \kappa.e : \forall a : \kappa . \sigma}
\and
\inference[AppT]
  {\Gamma \ve e : \forall a : \kappa .\sigma
   & \Gamma \vk \kappa : \delta
   & \Gamma \vdash_{\delta} \varphi : \kappa}
  {\Gamma \ve e ~ \varphi : \sigma [\varphi/a]}
\\
\fbox{$\Gamma \vp p \to e : \sigma \to \tau$}
\\
\inference[Alt]
  {K : \forall \overline{a : \kappa} . \forall \overline{b : \iota}.
  \overline{\sigma} \to T ~ \overline{a} \in \Gamma
   & \theta = [\overline{v/a}]
   & \Gamma , \overline{b:\theta( \iota ) , \overline{x: \theta(\sigma)}} \ve e : \tau}
  {\Gamma \vp K ~\overline{b:\theta(\iota)} ~ \overline{x:\theta(\sigma)}
  \to e : T ~ \overline{v} \to \tau}
  \end{mathpar}
  \caption{Typing rules for the core language of $F_C(X)$}
  \label{fig:typerules}
\end{figure}

\begin{figure}
\begin{minipage}{0.7\linewidth}
\begin{mathpar}
\fbox{$\Gamma \vdash decl : \Gamma'$}
\\
\inference[Data]
  {\overline{\Gamma \vty \sigma : \star}
   & \Gamma \vk \kappa : TY}
  {\Gamma \vdash \mathbf{data} ~ T:\kappa ~ \mathbf{where} 
    ~ \overline{K : \sigma} : (T : \kappa, \overline{K : \sigma})}
\\
\inference[Type]
  {\Gamma \vk \kappa : TY}
  {\Gamma \vdash (\mathbf{type} ~ S : \kappa) : (S : \kappa)}
\and
\inference[Coerce]
  {\Gamma \vk \kappa : CO}
  {\Gamma \vdash (\mathbf{axiom} ~ C : \kappa) : (C : \kappa)}
\end{mathpar}
\end{minipage}
\begin{minipage}{0.3\linewidth}
\begin{mathpar}
\fbox{$\Gamma \vdash pgm : \sigma$}
\\
\inference[Pgm]
  {\overline{\Gamma \vdash cdecl : \Gamma_c} & \Gamma =
  \Gamma_0,\overline{\Gamma_c}
  \\
  \overline{\Gamma \vdash decl : \Gamma_d} & \Gamma =
  \Gamma,\overline{\Gamma_d}
  \\
  \Gamma \ve e : \sigma}
  {\Gamma_0 \vdash \overline{cdecl}; \overline{decl}; e : \sigma}
\end{mathpar}
\end{minipage}
\caption{Environment extension rules for $F_C(X)$}
\label{fig:envextension}
\end{figure}
\subsection{Extended typing rules}
{\bf TODO:}
\begin{itemize}
  \item {\bf Explain what cast extrusion is and why it's bad}
  \item Need to formalize ``Get all the data constructors for this type
  constructor'' in this system since we will need it for ExtData in our
  environment extension rules.
  \item We need the extra equivalences since we need to prevent casts
  between differenet constructors that have the same type.
\end{itemize}
We now extend the core typing rules for System $F_C$ with a way to cast
between extended and core data constructors, while at the same time
preventing ``cast extrusion'' on the constructors and corresponding types.
\\

\noindent {\bf Definition}
We say that $\sigma_1 \tstack{\equiv}{[T_1,T_2]} \sigma_2$ if:
\begin{mathpar}
\sigma_1 = \forall \overline{a:\kappa} \forall
\overline{b:\iota}.\overline{\sigma}\to T_1 ~\overline{a}
\\
\textrm{and}
\\
\sigma_2 = \forall \overline{a:\kappa} \forall
\overline{b:\iota}.\overline{\sigma}\to T_2 ~\overline{a}
\end{mathpar}
i.e., that they are they same type modulo substitution of the type
constructor.
\\

\noindent {\bf Definition:} We define $\zeta\llbracket T/T' \rrbracket$ to
be the substitution of the type constructor application the underlying type
$\sigma$.
\\

\noindent{\bf Definition:} We say that $T$ extends $T'$ and that $T > T'$ if $T$
is a type constructor derived using a {\bf extends} form.

\begin{thm}[Transitivity of $(>)$]
\begin{proof}
WRITE ME \qedhere
\end{proof}
\end{thm}

\begin{thm}[$\llbracket T/T' \rrbracket$ preserves well-typedness]
\begin{proof}
WRITE ME \qedhere
\end{proof}
\end{thm}

\begin{thm}[Extension casts preserve well-typedness]
\begin{proof}
WRITE ME \qedhere
\end{proof}
\end{thm}

\begin{thm}[Domain-restricted reachability of types]
\begin{proof}
WRITE ME \qedhere
\end{proof}
\end{thm}

We then extend the typing judgements in Figure~\ref{fig:typerules} with the
two rules in Figure~\ref{fig:typerulesextra}, and omit the other rules (transitivity etc.) since these can be
easily figured out.

\begin{figure}[H]
\begin{mathpar}
\inference[ECoreCast]
  {\Gamma \ve e : \zeta'
   & \Gamma \vco \gamma : \zeta' \tstack{\sim}{[T_1, T_2]} \zeta}
  {\Gamma \ve e \tstack{\triangleright}{[T_1,T_2]} \gamma: \zeta}
\and
\inference[CoCoreVar]
  {\zeta_1 = ({\tt K}, \sigma_1) \in \Gamma
   & \zeta_2 = ({\tt K}, \sigma_2) \in \Gamma
   \\
   \exists T_1, T_2 .\left( \sigma_1 \tstack{\equiv}{[T_1, T_2]} \sigma_2\right)
   &
   (\gamma : \zeta_1 \tstack{\sim}{[T_1,T_2]} \zeta_2) \in \Gamma
   }
  {\Gamma \vco \gamma : \zeta_1 \tstack{\sim}{[T_1, T_2]} \zeta_2}
\\
\fbox{$\Gamma \vdash cdecl : \Gamma'$}
\\
\inference[CoreData]
  {\overline{\Gamma \vty \sigma : \star}
   & \Gamma \vk \kappa : TY}
  {\Gamma \vdash \mathbf{data}\;\mathbf{open}\;{T\; : \;
  \overline{\kappa} \;\mathbf{where}\;} \overline{K : \sigma} :
  (T : \kappa, \overline{K : \sigma}, (T;\overline{(K , \sigma)}))}
\and
\inference[ExtData]
  {\overline{\Gamma \vty \sigma : \star}
   & \Gamma \vk \kappa : TY
   \\
   \overline{(K',\sigma')} \stackrel{\Delta}{=}\overline{\forall (K',
   \sigma'') \in lookup(T,\Gamma).(K',\sigma'')\llbracket T'/T \rrbracket}
   }
  {\Gamma \vdash \mathbf{data}\;{T\; : \;
  \overline{\kappa} \;\mathbf{where}\;} \overline{K : \sigma}\;
  \mathbf{extends}(\overline{T}) :
  (T':\kappa, \overline{K:\sigma}, \overline{K':\sigma'}, 
  \overline{(K',\sigma') \tstack{\sim}{[T, T']} (K',\sigma'')})
  }
\end{mathpar}
\caption{Extension of the typing rules to allow extension of data types}
\label{fig:typerulesextra}
\end{figure}

\subsubsection{Insertion of Casts}

In order to insert the casts correctly, we must do coverage analysis of the
matching on the data types e.g.,
\begin{lstlisting}
f :: T1 -> T2
f ...
\end{lstlisting}
In this case we perform coverage analysis on @T1@ and, when @f@ fails to
match on all of the data types, automatically inserts a default clause (if
there is not already one) which behaves like the identity function, except
that it inserts a cast $(K,\sigma) \tstack{\triangleright}{T_1,T_2}$. We
now formalise this argument.

Define the set of all constructors of an extended data type $T$ 
coming from extended data types as $C$. Define the set of all specific
constructors defined by $T$ as $D$. Then we have that all of the
constructors of $T$ are $C \squarecup D$.

Now when performing coverage analysis of @f@ we consider two cases --
letting $R$ be the covered cases and $R'$ the non-covered cases:
\begin{itemize}
  \item $R = C \squarecup D$: In this case, the programmer has manually
    coded a conversion between the two data types. Therefore, we do not
    have to insert any casts.
  \item $R \supsetneq D$. Then we add a default clause to @f@ that performs
    a type coercion on the input.
  \item $R \subsetneq D$. In this case, it is an error, and we do not
    insert casts, since we cannot add a catchall clause to @f@.\note{It is
    worth noting that we could relax this requirement and add patterns for
  everything in $C \setminus R$ this would provide correctness right?}
\end{itemize}































































